\documentclass[class=article, crop=false]{standalone}
\usepackage[subpreambles=true]{standalone}
\usepackage{import}
\usepackage[T1]{fontenc}
\usepackage[utf8]{inputenc}
\usepackage[english, danish]{babel}
\usepackage[shortlabels]{enumitem}
% Initializing the counters and define a custom label
\newcommand{\reqinit}{
% Create a new counter for keeping track of the last number
\newcounter{reqcountbackup}
% Create a new counter for the custom label
\newcounter{reqcount}
% Redefine the command for the last counter so when it is called
% it prints the number like this in a bold font: R<number>
\renewcommand{\thereqcount}{\textbf{K\arabic{reqcount}}}
}

% Used to define the start of the requirements
\newcommand{\reqstart}{
% Indicate the start of a new list and tell it to use the redefined
% command and corresponding counter for every item
\begin{list}{\thereqcount}{\usecounter{reqcount}}
% Important part: set the value of the used counter to the
% same value of the backup counter.
\setcounter{reqcount}{\value{reqcountbackup}}
}

% Used to define the end of the requirements
\newcommand{\reqend}{
% Important part: take the value of the used counter (after
% being incremented by the requirement items) and store it
% in the backup counter.
\setcounter{reqcountbackup}{\value{reqcount}}
% Mark the end of the list environment
\end{list}
}





\begin{document}
\reqinit
\paragraph{Prioriteret funktional-krav liste}
\subparagraph{Høj prioritet}
\reqstart
\item Der skal kunne være 3-6 spillere.
\item Spiller kunne få 30.000kr. i starten af spillet.
\item Spiller skal kunne starte på 'Start' felt.
\item Spiller skal kunne få 4.000kr, hver gang spiller kommer til, eller passerer 'Start'.
\item Spiller skal kunne kaste terning og flytte sin brik så mange felter frem (venstre om), som øjnene viser.
\item Spiller skal kunne købe en grund efter prisen, som står på feltet hver gang spiller lander på en grund som ikke er købt.
\item Spiller skal kunne betale leje til ejeren af feltet hver gang spiller lande på grund ejes af en anden spiller.
\item Spiller bliver flyttet fra 'Til Fængsel' til 'Fængsel' hver gang spiller lander på 'Til Fængsel'. Spilleren modtager altså ikke 4.000kr.
\item Standser man på 'Fængsel' felt, er man imidlertid blot på besøg og fortsætter næste gang uden straf.
\item Spiller kan komme ud af fængsel ved at betale en bøde på 1.000kroner inden man kaster
\item Parkering er et fristed til man skal kaste igen.
\item Spiller taber, når spiller skylder mere end han ejer.
\reqend
\subparagraph{Mellem prioritet}
\reqstart
\item Spiller skal kunne trække ét chancekort når der landes på feltet “Chance”
\item Spiller skal kunne komme ud af fængsel ved at benytte et af løsladelseskortene fra lykkekortene.
\item Spiller skal kunne komme ud af fængsel ved at kaste 2 af samme slags.
\reqend
\subparagraph{Lav prioritet}
\reqstart
\item Banken skal kunne sætte et hus til auktion, hvis en spiller ikke vil købe det.
\item Spiller skal kunne pantsætte grunde til banken for at skaffe flere penge.
\item Lejesummen skal kunne forøges betydeligt ved opførelse af hus og hotteler.
\item Ekstra kast får man, hvis man kaster 2 af samme slags (f. eks. 2 femmere), og man retter sig både efter forskrifterne for det felt, man kommer til efter første kast, og efter ekstrakastet.
\item Kaster man 3 gange i træk 2 af samme slags, må man ikke flytte tredie gang, men skal gå direkte i fængsel.
\item Spiller kan ikke blive i fængsel mere end tre omgange. Får spiller ikke to af samme slags, når man kaster tredie gang, må spiller betale bøden, 1.000 kroner, og flytte, som øjnene viser.
\item Spiller skal kunne bygge ét hotel på hver grund, når spiller har fire huse på hver grund i gruppen. Og aflevere de fire huse til banken.
\item Spiller skal kunne sægle huse til banken til halv pris, når som helst man ønsker
\item Der skal bygges jævnt dvs. det første hus kan man opføre på hvilken grund i gruppen, man ønsker; men inden hus nr. 2 opføres på en grund, skal der være bygget eet på hver af de andre grunde i gruppen osv.
\reqend


\paragraph{Supplerende specifikationer}

\subparagraph{Supportability}
\reqstart
\item Kan ændres til andre sprog (Lav prioritet)
\reqend

\subparagraph{Extensibility}
\reqstart
\item Skal overholdes GRASP (Højt prioritet)
\reqend

\subparagraph{Hardware}
\reqstart
\item Spillet skal virke på maskinerne i DTU's databarere (Højt prioritet)
\reqend

\subparagraph{Software}
\reqstart
\item Skal virke på Java version 8(Højt prioritet)
\reqend

\end{document}