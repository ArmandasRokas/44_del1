\documentclass[class=article, crop=false]{standalone}
\usepackage[subpreambles=true]{standalone}
\usepackage{import}
\usepackage[T1]{fontenc}
\usepackage[utf8]{inputenc}
\usepackage[english, danish]{babel}
\usepackage[shortlabels]{enumitem}
% Initializing the counters and define a custom label
\newcommand{\reqinit}{
% Create a new counter for keeping track of the last number
\newcounter{reqcountbackup}
% Create a new counter for the custom label
\newcounter{reqcount}
% Redefine the command for the last counter so when it is called
% it prints the number like this in a bold font: R<number>
\renewcommand{\thereqcount}{\textbf{K\arabic{reqcount}}}
}

% Used to define the start of the requirements
\newcommand{\reqstart}{
% Indicate the start of a new list and tell it to use the redefined
% command and corresponding counter for every item
\begin{list}{\thereqcount}{\usecounter{reqcount}}
% Important part: set the value of the used counter to the
% same value of the backup counter.
\setcounter{reqcount}{\value{reqcountbackup}}
}

% Used to define the end of the requirements
\newcommand{\reqend}{
% Important part: take the value of the used counter (after
% being incremented by the requirement items) and store it
% in the backup counter.
\setcounter{reqcountbackup}{\value{reqcount}}
% Mark the end of the list environment
\end{list}
}





\begin{document}
\reqinit
\paragraph{Prioriteret funktional-krav liste \cite{spilleregler}}
\subparagraph{Høj prioritet}
\reqstart
\item Der skal kunne være 3-6 spillere.
\item Spiller kunne få 30.000kr. i starten af spillet.
\item Spiller skal kunne starte på 'Start' felt.
\item Spiller skal kunne få 4.000kr, hver gang spiller kommer til, eller passerer 'Start'.
\item Spiller skal kunne kaste terning og flytte sin brik så mange felter frem (venstre om), som øjnene viser.
\item Spiller skal kunne købe en grund efter prisen, som står på feltet hver gang spiller lander på en grund som ikke ejes af nogen.
\item Spiller skal kunne betale leje til ejeren af feltet hver gang spiller lande på grund ejes af en anden spiller.
\item Spiller bliver flyttet fra 'Til Fængsel' til 'Fængsel' hver gang spiller lander på 'Til Fængsel'. Spilleren modtager yderligere ikke 4.000kr.
\item Spiller skal kunne fortsætte uden straf, når spiller stander på 'Fængsel'.
\item Spiller skal kunne komme ud af fængsel ved at betale en bøde på 1.000kroner inden spiller kaster.
\item Spiller skal kunne standse på 'Parkering' gratis og fortsætte derfra til næste kast.
\item Spiller taber, når spiller skylder mere end han ejer.
\reqend
\subparagraph{Mellem prioritet}
\reqstart
\item Spiller skal kunne trække ét chancekort når der landes på feltet “Chance”
\item Spiller skal kunne komme ud af fængsel ved at benytte et af løsladelseskortene fra lykkekortene.
\item Spiller skal kunne komme ud af fængsel ved at kaste 2 af samme slags. Spiller skal samtidig kunne få en ekstra tur.
\item Spiller skal kunne få en ekstra tur, hvis man kaster 2 af samme slags.
\item Spiller skal kunne gå direkte i fængsel, hvis spiller kaster 3 gang i træk af samme slags.
\reqend
\subparagraph{Lav prioritet}
\reqstart
\item Spillen skal indholde 32 grønne huse og 12 røde hoteller, hvilket betyder, at hvis banken ikke har nogen bygninger, når spiller vil købe, må spiller vente, til der kommer nogle tilbage. Er der flere, der vil købe, og banken ikke har nok bygninger, sætter den dem, der er, til auktion.
\item Banken skal kunne sætte et hus til auktion, hvis en spiller ikke vil købe det.
\item Spiller skal kunne pantsætte grunde til banken for at skaffe flere penge. Hvis spiller har bygninger på grundene, skal spiller først sælge disse til banken. Af pansat ejendom kan der ikke kræves leje. Renten er 10proc, der betales sammen med lånet, når pantsætningen hæves.
\item Spiller skal bygge jævnt dvs. det første hus kan spiller opføre på hvilken grund i gruppen, spiller ønsker. Men inden hus nr. 2 opføres på en grund, skal der være bygget ét på hver af de andre grunde i gruppen osv.
\item Hvis en pantsat ejendom sælges, og køberen ikke straks hæver pantsætningen, må
han alligevel betale 10 proc. Hvis han senere hæver pantsætningen, atter 10.
\item Lejesummen skal kunne forøges betydeligt ved opførelse af hus og hotteler.
\item Spiller skal ikke kunne blive i fængsel mere end tre omgange. Får spiller ikke to af samme slags, når man kaster tredje gang, må spiller betale bøden, 1.000 kroner, og flytte, som øjnene viser.
\item Spiller skal kunne bygge ét hotel, når spiller har fire huse på hver grund i gruppen. Spiller skal aflevere disse fire huse til banken. Spiller må kun bygge ét hotel på hver grund. Prisen for et hotel er fem gange prisen for et hus.
\item Spiller skal kunne sægle huse til banken til halv pris, når som helst spiller ønsker
\item Spiller får udleveret skødet, når spiller køber en grund. Skødet skal være synlig for alle spiller.
\item Den fængslede spiller har ret til at købe grunde.
\item Ejer spiller alle grundene i samme farve, får man dobbelt leje af de ubebyggede grunde og har ret til når som helst at bygge huse, der købes hos banken.
\item Når spiller går konkurs, skal spiller overdrage alt til sin kreditor efter at have solgt eventuelle bygninger tilbage til banken. Er det banken, der er kreditor, sælger han straks modtagne grunde ved auktion.
\reqend


\paragraph{Supplerende specifikationer}

\subparagraph{Supportability}
\reqstart
\item Kan ændres til andre sprog (Lav prioritet)
\reqend

\subparagraph{Extensibility}
\reqstart
\item Skal overholdes GRASP (Højt prioritet)
\reqend

\subparagraph{Hardware}
\reqstart
\item Spillet skal virke på maskinerne i DTU's databarere (Højt prioritet)
\reqend

\subparagraph{Software}
\reqstart
\item Skal virke på Java version 8(Højt prioritet)
\reqend

\end{document}