\documentclass[class=article, crop=false]{standalone}
\usepackage[subpreambles=true]{standalone}
\usepackage{import}
\usepackage[T1]{fontenc}
\usepackage[utf8]{inputenc}
\usepackage[english, danish]{babel}
\usepackage[shortlabels]{enumitem}
% Initializing the counters and define a custom label
\newcommand{\reqinit}{
% Create a new counter for keeping track of the last number
\newcounter{reqcountbackup}
% Create a new counter for the custom label
\newcounter{reqcount}
% Redefine the command for the last counter so when it is called
% it prints the number like this in a bold font: R<number>
\renewcommand{\thereqcount}{\textbf{K\arabic{reqcount}}}
}

% Used to define the start of the requirements
\newcommand{\reqstart}{
% Indicate the start of a new list and tell it to use the redefined
% command and corresponding counter for every item
\begin{list}{\thereqcount}{\usecounter{reqcount}}
% Important part: set the value of the used counter to the
% same value of the backup counter.
\setcounter{reqcount}{\value{reqcountbackup}}
}

% Used to define the end of the requirements
\newcommand{\reqend}{
% Important part: take the value of the used counter (after
% being incremented by the requirement items) and store it
% in the backup counter.
\setcounter{reqcountbackup}{\value{reqcount}}
% Mark the end of the list environment
\end{list}
}





\begin{document}
\reqinit
\paragraph{Prioriteret funktional-krav liste \cite{spilleregler}}
\subparagraph{Must have}
\reqstart
\item Der skal kun være 3-6 spillere.
\item Spiller får 1500kr. i starten af spillet.
\item Spiller starter på 'Start' felt.
\item Spiller får 200kr, hver gang spiller kommer til, eller passerer 'Start'.
\item Spiller kaster terning og flytter sin brik så mange felter frem (med uret), som øjnene viser.
\item Hver spiller skal kunne købe en grund til prisen, som står på feltet, hver gang spilleren lander på en grund, som ikke ejes af nogen.
\item Når en spiller lander på grund, der ejes af en anden spiller, skal spilleren betale leje til ejeren af feltet.
\item Når en spiller lander på 'Til Fængsel', rykkes der videre til 'Fængsel'. Spilleren modtager yderligere ikke 200kr, når 'Start' passeres.
\item Hvis en spiller lander på 'Fængsel', skal spilleren kunne fortsætte uden straf.
\item Spiller kan komme ud af fængslet ved at betale en bøde på 50kr, inden spiller kaster.
\item Spiller skal kunne lande på 'Helle' og fortsætter derfra ved næste kast.
\item Når en spiller skylder mere end han ejer, går spilleren fallit.
\item Spiller har ret til når som helst at bygge huse, når spiller ejer alle grundene af samme farve. Husene skal bygges jævnt dvs. det første hus kan spiller opføre på hvilken grund i gruppen, spiller ønsker. Men inden hus nr. 2 opføres på en grund, skal der være bygget ét på hver af de andre grunde i gruppen osv.
\item Spiller skal kunne bygge ét hotel, når spiller har fire huse på hver grund i samme farve. Spilleren skal aflevere disse fire huse til banken. Spiller må kun bygge ét hotel på hver grund. Prisen for et hotel er det samme som for et hus.
\item Lejesummen forøges betydeligt ved opførelse af hus og hotteler.
\item Ejer spiller alle grundene i samme farve, får man dobbelt leje af de ubebyggede grunde
\item Når spiller går fallit, skal spiller overdrage alt til banken.
\reqend
\subparagraph{Should have}
\reqstart
\item Spiller skal kunne få en ekstra tur, hvis man kaster 2 af samme slags.
\item Spiller skal kunne trække ét chancekort når der landes på feltet 'Chance'
\item Spiller skal kunne komme ud af fængslet ved at benytte et løsladelseskort fra chancekortene.
\item Spiller skal kunne komme ud af fængslet ved at kaste 2 af samme slags, og spilleren skal samtidig få en ekstra tur.
\item Spiller kan ikke blive i fængsel mere end tre omgange. Får spilleren ikke to af samme slags, når der kastes tredje gang, må spilleren betale en bøde på 50kr, og flytte, som øjnene viser.
\item Spiller skal rykkes direkte i fængslet, hvis spiller kaster 3 gang i træk af samme slags.
\item Når en spiller lander på 'Indkomstskatten', skal spilleren betale 200kr eller 10\% . af sine værdier: Kontanter, bygninger og den trykke pris for grunde og virksomheder(også pantsatte). Spilleren skal vælge betalingsmåde, inden spiller kender sine værdier.
\item Når en spiller lander på 'Ekstraordinær statsskat', skal spilleren betale 100kr.
\reqend
\subparagraph{Could have}
\reqstart
\item Banken skal kunne sætte en grund til auktion, hvis en spiller ikke vil købe den.
\item Spiller skal kunne pantsætte en eller flere grunde til banken for at modtage pantsætningsværdien. Hvis spiller har bygninger på grundene, skal spilleren først sælge disse til banken. Af pansat ejendom kan der ikke kræves leje. Renten er 10\%, der betales sammen med lånet, når pantsætningen hæves.
\item Hvis en pantsat ejendom sælges, og køberen ikke straks hæver pantsætningen, skal han alligevel betale 10\%. Hvis han senere hæver pantsætningen, skal spilleren betale 10\% oven i ophævningsprisen som normalt.
\item Spiller skal kunne sægle huse til banken til halv pris, når som helst spiller ønsker
\item Spiller får udleveret skødet, når spiller køber en grund. Skødet skal være synlig for alle spiller.
\item Den fængslede spiller har ret til at købe grunde.
\item Når spiller går fallit, skal spiller overdrage alt til sin kreditor efter at have solgt eventuelle bygninger tilbage til banken. Er det banken, der er kreditor, sælger han straks modtagne grunde ved auktion.
\reqend
\subparagraph{Wont have}
\reqstart
\item Spillen skal indholde 32 grønne huse og 12 røde hoteller, hvilket betyder, at hvis banken ikke har nogen bygninger, når spiller vil købe, må spiller vente, til der kommer nogle tilbage. Er der flere, der vil købe, og banken ikke har nok bygninger, sætter den dem, der er, til auktion.
\reqend

\paragraph{Supplerende specifikationer}

\subparagraph{Supportability}
\reqstart
\item Kan ændres til andre sprog (Lav prioritet)
\reqend

\subparagraph{Extensibility}
\reqstart
\item Skal overholdes GRASP (Højt prioritet)
\reqend

\subparagraph{Hardware}
\reqstart
\item Spillet skal virke på maskinerne i DTU's databarere (Højt prioritet)
\reqend

\subparagraph{Software}
\reqstart
\item Skal virke på Java version 8(Højt prioritet)
\reqend

\end{document}