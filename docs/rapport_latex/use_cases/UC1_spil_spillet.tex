\documentclass[class=article, crop=false]{standalone}
\usepackage[subpreambles=true]{standalone}
\usepackage{import}
\usepackage[T1]{fontenc}
\usepackage[utf8]{inputenc}
\usepackage[english, danish]{babel}
\usepackage{tgbonum}
\usepackage{tabularx}
\usepackage{booktabs}
\usepackage{enumitem}
\newlist{tabenum}{enumerate}{1}
\setlist[tabenum]{label=\arabic*. ,wide=0pt, leftmargin=*, nosep, itemsep=2pt, font = \bfseries, after=\vspace{-\baselineskip}, before=\compress}
\usepackage{array, booktabs}
\makeatletter
\newcommand*{\compress}{\@minipagetrue}
\makeatother
\usepackage{paralist}
\makeatletter
\let\savespace\@minipagetrue
\makeatother



% Document
\begin{document}

    \begin{table}[H]
        \caption{Full-dressed use case. UC1: Spil spillet}
        \begin{tabularx}{\textwidth}{|l|X|}
            \hline
                                         & \textbf{[UC1]: Spil Spillet}   \\ \hline
            \textbf{Scope}               & Matador Applikation\\ \hline
            \textbf{Level}               & User goal     \\ \hline
            \textbf{Primær aktør}        & 3-6 spillere  \\ \hline
            \textbf{Interessenter}       & IOOuterActive\\ \hline
            \textbf{Preconditions}       & Spillet er startet \\ \hline
            \textbf{Postconditions}      & Kun én spiller tilbage i spillet,
                                           som udpeges som en vinder\\ \hline





            \textbf{Basic Flow} & \begin{tabenum}
                          \item Spiller indtaster antal spillere
                          \item Spiller indtaster navn
                              \savespace
                                \begin{compactitem}
                                    \item \textit{Spiller gentager trin 2                                                             indtil alle har et navn}
                          \end{compactitem}
                              \item Spiller kaster terningerne
                              \item Systemet viser resultat af kastet, rykker spilleren og viser scenariet for feltet man er lander på.
                              \item Spiller agerer på scenariet
                              \item Systemet udfører scenariet, kontrollerer for om, der en spiller, der gået fallit.
                          \item Systemet giver turen til næste spiller, hvis spilleren ikke har slået to ens.
                          \savespace
                          \begin{compactitem}
                              \item \textit{Gentages trin 3-6, indtil der er én spiller tilbage i spillet}
                          \end{compactitem}
                              \item Afsluttet spil og udskriver resultat.
                           \end{tabenum}   \\ \hline




    \textbf{Alternate Flow}   & \textbf{3.a.} Spiller er i 'Fængsel' \textit{(ikke på besøg)}.
    \begin{enumerate} \begin{tabenum}
                          \item Jf. UC3: \underline{Kom ud af fængsel}
    \end{tabenum} \end{enumerate}
    \\



                              & \textbf{4.a.} Spiller har passeret start.
                                \begin{enumerate} \begin{tabenum}
                                        \item Systemet tilføjer 200kr til spillerens konto.
                                        \item Udføres trin 4-7 som sædvanlig
                                    \end{tabenum} \end{enumerate}
                                 \\


                            & \textbf{4.b.} Spiller har kastet 2 af samme slags 3. gang i træk.
                            \begin{enumerate} \begin{tabenum}

                                                  \item Systemet rykker spilleren i fængsel og skifter tur til næste spiller, som fortsætter fra trin 3.
                            \end{tabenum} \end{enumerate}
                            \\




                              & \textbf{5.a.a.} Spiller er landet på feltet
                                'Grund', som ejes af en anden spiller.
                                \begin{enumerate} \begin{tabenum}
                                  \item Spiller betaler leje til kreditoren
                                        jf. skøderne i Bilag \#.
                                  \item Fortsættes fra trin 6.
                                \end{tabenum} \end{enumerate}
                                \\
                            & \textbf{5.a.b.} Spiller lander på feltet
                                'Grund', som ikke ejes af nogen.
                            \begin{enumerate} \begin{tabenum}
                                  \item Jf. UC2: \underline{Køb grund}.
                                  \item Fortsættes fra trin 6
                            \end{tabenum} \end{enumerate}
                            \\


                            & \textbf{5.b.} Spiller lander på feltet
                            'Indkomstskatten'.
                            \begin{enumerate} \begin{tabenum}
                              \item Spiller vælger betalingsmåde(10\% af sine værdierne eller 200kr).
                              \item Fortsættes fra trin 6.
                            \end{tabenum} \end{enumerate}
                            \\


                            & \textbf{5.c.} Spiller lander på feltet 'Parkering'/'Fængsel'/'Start'
                            \begin{enumerate} \begin{tabenum}
                                                  \item Fortsættes fra trin 7.
                            \end{tabenum} \end{enumerate}
                            \\


                            & \textbf{5.d.} Spiller lander på feltet 'Til Fængsel'
                            \begin{enumerate} \begin{tabenum}
                                                  \item Systemet rykker spiller til feltet 'Fængsel' uden at tilføje 200kr til spilleren's konto.
                                                  \item Fortsættes fra trin 7.
                            \end{tabenum} \end{enumerate}
                            \\


                            & \textbf{5.e.} Spiller lander på feltet 'Prøv lykken'
                            \begin{enumerate} \begin{tabenum}
                                                  \item Jf. UC4: \underline{Prøv lykken}
                            \end{tabenum} \end{enumerate}
                            \\

                            & \textbf{6.a.} Spiller går fallit
                            \begin{enumerate} \begin{tabenum}
                                                  \item Jf. UC10: \underline{Gå fallit}
                            \end{tabenum} \end{enumerate}
                            \\


            \hline








            \textbf{Teknologi og}     & Java 8LTS: JRE \\
            \textbf{Data variantions} &  \\ \hline
            \textbf{Kørselsfrekvens} & 1 gang pr. spil\\ \hline
        \end{tabularx}


    \end{table}

\end{document}


