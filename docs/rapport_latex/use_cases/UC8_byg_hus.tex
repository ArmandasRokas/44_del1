
\documentclass[class=article, crop=false]{standalone}
\usepackage[subpreambles=true]{standalone}
\usepackage{import}
\usepackage[T1]{fontenc}
\usepackage[utf8]{inputenc}
\usepackage[english, danish]{babel}
\usepackage{tgbonum}
\usepackage{tabularx}
\usepackage{booktabs}
\usepackage{enumitem}
\newlist{tabenum}{enumerate}{1}
\setlist[tabenum]{label=\arabic*. ,wide=0pt, leftmargin=*, nosep, itemsep=2pt, font = \bfseries, after=\vspace{-\baselineskip}, before=\compress}
\usepackage{array, booktabs}
\makeatletter
\newcommand*{\compress}{\@minipagetrue}
\makeatother
\usepackage{paralist}
\makeatletter
\let\savespace\@minipagetrue
\makeatother



% Document
\begin{document}

    \begin{table}[H]
        \caption{Use case. UC8: Byg hus}
        \begin{tabularx}{\textwidth}{|l|X|}
            \hline
            & \textbf{[UC8]: Byg hus}   \\ \hline
            \textbf{Preconditions}       & Spiller ejer alle grunde i samme farve, har penge nok.\\ \hline
            \textbf{Postconditions}      & Huset er bygget på spilleren's grund\\ \hline


            \textbf{Basic Flow} & \begin{tabenum}
                                      \item Spiller vælger til at bygge en bygning
                                      \item Systemet viser grunde, hvor spiller må bygge et hus.
                                      \item Spiller vælger en grund.
                                      \item Systemet fjerner et beløb jf. skøderne fra spillerens konto, tilføjer huset til grunden og øger lejeprisen.
            \end{tabenum}   \\ \hline

        \end{tabularx}


    \end{table}

\end{document}

