
\documentclass[class=article, crop=false]{standalone}
\usepackage[subpreambles=true]{standalone}
\usepackage{import}
\usepackage[T1]{fontenc}
\usepackage[utf8]{inputenc}
\usepackage[english, danish]{babel}
\usepackage{tgbonum}
\usepackage{tabularx}
\usepackage{booktabs}
\usepackage{enumitem}
\newlist{tabenum}{enumerate}{1}
\setlist[tabenum]{label=\arabic*. ,wide=0pt, leftmargin=*, nosep, itemsep=2pt, font = \bfseries, after=\vspace{-\baselineskip}, before=\compress}
\usepackage{array, booktabs}
\makeatletter
\newcommand*{\compress}{\@minipagetrue}
\makeatother
\usepackage{paralist}
\makeatletter
\let\savespace\@minipagetrue
\makeatother



% Document
\begin{document}

    \begin{table}[H]
        \caption{Use case. UC5: Pantsætte grund}
        \begin{tabularx}{\textwidth}{|l|X|}
            \hline
            & \textbf{[UC4]: Prøv lykken}   \\ \hline
            \textbf{Preconditions}       & Spiller har en ubebyggede grund\\ \hline
            \textbf{Postconditions}      & Spiller har pantsat sin grund\\ \hline


            \textbf{Basic Flow} & \begin{tabenum}
                                      \item Spiller trækker et tilfældigt lykkekort
                                      \item Systemet viser scenariet for lykkekortet.
                                      \item Spiller agere på scenariet.
            \end{tabenum}   \\ \hline




            \textbf{Alternate Flow}   & \textbf{3.a.} Spiller har trukket lykkekortet ' ' .
            \begin{enumerate} \begin{tabenum}
                                  \item a
            \end{tabenum} \end{enumerate}
            \\



            \hline

        \end{tabularx}


    \end{table}

\end{document}
